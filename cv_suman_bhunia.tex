%% start of file `template.tex'.
%% Copyright 2006-2012 Xavier Danaux (xdanaux@gmail.com).
%
% This work may be distributed and/or modified under the
% conditions of the LaTeX Project Public License version 1.3c,
% available at http://www.latex-project.org/lppl/.


\documentclass[11pt,letterpaper,sans]{moderncv}   % possible options include font size ('10pt', '11pt' and '12pt'), paper size ('a4paper', 'letterpaper', 'a5paper', 'legalpaper', 'executivepaper' and 'landscape') and font family ('sans' and 'roman')

% moderncv themes
\moderncvstyle{classic}                        % style options are 'casual' (default), 'classic', 'oldstyle' and 'banking'
\moderncvcolor{blue}                          % color options 'blue' (default), 'orange', 'green', 'red', 'purple', 'grey' and 'black'
%\renewcommand{\familydefault}{\sfdefault}    % to set the default font; use '\sfdefault' for the default sans serif font, '\rmdefault' for the default roman one, or any tex font name
%\usepackage[scaled]{helvet}

%\nopagenumbers{}                             % uncomment to suppress automatic page numbering for CVs longer than one page

% character encoding
%\usepackage[utf8]{inputenc}                  % if you are not using xelatex ou lualatex, replace by the encoding you are using
%\usepackage{CJKutf8}                         % if you need to use CJK to typeset your resume in Chinese, Japanese or Korean

% adjust the page margins
\usepackage[top=0.75in, bottom=0.75in, left=1in, right=0.75in]{geometry}
\usepackage{enumerate}
%\setlength{\hintscolumnwidth}{3cm}           % if you want to change the width of the column with the dates
%\setlength{\makecvtitlenamewidth}{10cm}      % for the 'classic' style, if you want to force the width allocated to your name and avoid line breaks. be careful though, the length is normally calculated to avoid any overlap with your personal info; use this at your own typographical risks...

% personal data
\firstname{Suman}
\familyname{Bhunia}
%\title{Resumé title}                          % optional, remove / comment the line if not wanted
%\address{1050 Nevada Street, Apt 121}{Reno, NV 89503}    % optional, remove / comment the line if not wanted
%\mobile{+1~(551)~587~0783}                     % optional, remove / comment the line if not wanted
%\phone{+1~(513)~529~0339}                      % optional, remove / comment the line if not wanted
%\fax{+3~(456)~789~012}                        % optional, remove / comment the line if not wanted
\email{bhunias@miamioh.edu}                          % optional, remove / comment the line if not wanted
%\homepage{www.cse.unr.edu}                  % optional, remove / comment the line if not wanted
\extrainfo{www.sbhunia.me}            % optional, remove / comment the line if not wanted

%\quote{Some quote}                            % optional, remove / comment the line if not wanted

% to show numerical labels in the bibliography (default is to show no labels); only useful if you make citations in your resume
\makeatletter
\renewcommand*{\bibliographyitemlabel}{\@biblabel{\arabic{enumiv}}}
\makeatother

% bibliography with mutiple entries
\usepackage[resetlabels]{multibib}
\newcites{bibj,bibp,bibc,bibs}{{Journals},{Short\ Papers\ and\ Posters},{Conference\ Proceedings},{Short\ Papers\ and\ Posters}}
\usepackage{enumerate}
%----------------------------------------------------------------------------------
%            content
%----------------------------------------------------------------------------------



%\newcommand{\makeauthorbold}[1]{%
%	\DeclareNameFormat{author}{%
%		\ifthenelse{\value{listcount}=1}
%		{%
%			{\expandafter\ifstrequal\expandafter{\namepartfamily}{#1}{\mkbibbold{\namepartfamily\addcomma\addspace \namepartgiveni}}{\namepartfamily\addcomma\addspace \namepartgiveni}}
%			%
%		}{\ifnumless{\value{listcount}}{\value{liststop}}
%			{\expandafter\ifstrequal\expandafter{\namepartfamily}{#1}{\mkbibbold{\addcomma\addspace \namepartfamily\addcomma\addspace \namepartgiveni}}{\addcomma\addspace \namepartfamily\addcomma\addspace \namepartgiveni}}
%			{\expandafter\ifstrequal\expandafter{\namepartfamily}{#1}{\mkbibbold{\addcomma\addspace \namepartfamily\addcomma\addspace \namepartgiveni\addcomma\isdot}}{\addcomma\addspace \namepartfamily\addcomma\addspace \namepartgiveni\addcomma\isdot}}%
%		}
%		\ifthenelse{\value{listcount}<\value{liststop}}
%		{\addcomma\space}{}
%	}
%}
%\makeauthorbold{Bhunia}



\begin{document}
%\begin{CJK*}{UTF8}{gbsn}                     % to typeset your resume in Chinese using CJK
%-----       resume       ---------------------------------------------------------

% Making the title
\makecvtitle
\vspace{-3\baselineskip}

% This is for creating header on each page except the front page
\rhead{\ifthenelse{\value{page}=1}{}{\color{gray}Suman Bhunia~~ \em Curriculum Vitae}}



\section{Professional Experience}

\cventry{2020–current} {Assistant Professor} {Department of Computer Science and Software Engineering}{} {}{Miami University, Oxford, Ohio USA}
\cventry{2018–2020}{Postdoctoral Researcher}{Dept. of Computer Science and Engineering}{}{}{Texas A\&M University, College Station, USA}
\cventry{2017–2018}{Postdoctoral Scholar}{Department of Computer Science}{}{} {University of California, Davis, USA}
\cventry{2013--2017}{Research Assistant \& Instructor}{Department of Computer Science and Engineering}{}{}{University of Nevada, Reno, USA
}
\cventry{2011–2013}{Research Assistant \& Adjunct Lecturer}{Department of Computer Science}{}{}{City University of New York, New York, USA}
\cventry{2010–2011}{Senior Research Fellow}{Department of Electronics and Telecommunication Engineering}{}{}{ Jadavpur University, Kolkata, India}



\section{Education}
\cventry{2013--present}{Ph.D. in Computer Science}{University of Nevada, Reno, USA}{\emph{Thesis:} Defense Against Intelligent Attacker in Cognitive Radio Networks}{GPA:4.0/4.0} {Committee: Shamik Sengupta, Murat Yuksel, Mehmet Gunes, Sergiu Dascalu, Sankar Mukhopadhyay}

\cventry{2008--2010}{Masters of Technology in Distributed and Mobile Computing}{Jadavpur University, Kolkata, India}{\emph{Thesis:} Performance Evaluation of WiMAX Network in Aspect of Modulation and Coding Schemes and Hand-off using OPNET}{ GPA:8.39/10}{Committee: Dr. Iti Saha Misra and Dr. Salil K. Sanyal}

\cventry{2004--2008}{Bachelor of Technology in Electronics and Communication Engineering}{West Bengal University of Technology}{Kolkata, India}{GPA:8.36/10}{}


\section{Awards and Fellowships}
\cvitem{2018}	{Outstanding Thesis award from the Department of Computer Science, UNR}

\cvitem{2017}	{Outstanding Graduating Graduate Student, UNR (Awarded only one student in the University)}

\cvitem{2016}	{Outstanding graduate student of Department of Computer Science, UNR}

\cvitem{2011--2013}	{Two year City University of New york science fellowship}

\cvitem{2010}	{Paper entitled ”Study of OPNET and performance evaluation of WiMAX network under various terrain conditions in OPNET” won the best student paper award at the National
Conference on Microwave and Communication NCMicroCom-2010}

\cvitem{2008--2010}{Two year full scholarship for M.Tech programme for qualifying Graduate Aptitude Test in 	Engineering, GATE (All-India basis)}


\section{Research Interest}
\cvitem{Security}{Network security, malicious node sensing/detection, cyber-physical security, wireless honeypot, jamming attack, spectrum fingerprinting, IoT Security}

\cvitem{Wireless}{6G, Cognitive radio, dynamic spectrum access (DSA), cross-layer optimization, ad hoc, unmanned autonomous systems (UAS)}

\cvitem{Networking}{QoS and resource management, Distributed Edge Computing, end-to-end performance, testbed implementation}

\section{Courses Taught }

\cvitem{CSE 470/570}{Ethical Hacking (at Miami University)
}
\cvitem{CSE 467/567}{Computer and Network Security (at Miami University)
}
\cvitem{CSE 271}{Object-Oriented Programming using Java (at Miami University)
}
\cvitem{CSE 274}{Data Abstraction and Data Structures (at Miami University)
}

\cvitem{CS 446/646}{Principles of Computer Operating Systems, (at University of Nevada, Reno)
}
\cvitem{CS 201}{Digital Design Laboratory (at University of Nevada, Reno)
}

\cvitem{CS 332}{Operating System Laboratory (at City College of New York)}
\cvitem{CS 102}{ Introduction to Programming (at City College of New York)}


% Publications from a BibTeX file without multibib
%  for numerical labels: \renewcommand{\bibliographyitemlabel}{\@biblabel{\arabic{enumiv}}}
%  to redefine the heading string ("Publications"): \renewcommand{\refname}{Articles}

% Publications from a BibTeX files using the multibib package
\section{Patent Invention Disclosure}
\cvitem{[1]}{M. R. Khan,  M. Yuksel, S. Bhunia, and S. Sengupta,  \textit{In-Band Line-of-Sight Discovery for Directional Full-Duplex Transceivers}, U.S. Provisional Patent Application 62/338,953}


\section{Peer Reviewed Publications}
\nocitebibj{*}
\bibliographystylebibj{IEEEtran}
\bibliographybibj{journal}              % 'publications' is the name of a BibTeX file


\nocitebibc{*}
\bibliographystylebibc{IEEEtran}
\bibliographybibc{conference}              % 'publications' is the name of a BibTeX file


\nocitebibp{*}
\bibliographystylebibp{IEEEtran}
\bibliographybibp{shortPapers}              % 'publications' is the name of a BibTeX file


\section{Research Grant Proposal Preparation}
\cvitem{TAMU}{	Prepared a part of the proposal, “Deploying Defenses for Cellular Networks Using the AWARE
Testbed”, submitted to DHS in 2019}
\cvitem{TAMU}{Prepared a part of the proposal, “DAAR: Drone-Augmented Augmented Reality for CyberHuman Physical Systems”, submitted to NSF CPS in 2019}
\cvitem{UC Davis}{	Prepared a part of the proposal, “Building Low-Power Wide Area Networking Systems for Smart City IoT Applications”, submitted to NSF ICE-T in 2018
}
\cvitem{UNR}{ Helped my advisor writing in a proposal, “Efficient Spectrum Access Utilizing Unmanned Autonomous Systems”, submitted to NSF EARS in 2014
}










\section{Research Projects}
\cventry{2018-2020}{Secured Edge Computing for Disaster Response Network}{Postdoctoral research}{}{}
{
	\begin{itemize}
		\item[--]Design EdgeKeeper- A distributed coordination scheme for Edge Network to facilitate Security, Naming and Service discovery in Opportunistic network.
		\item[--] Implementing EdgeKeeper using Java and Android that utilizes Zookeeper, GNS, etc.
	\end{itemize}
}

\cventry{2017--2018}{Security in cyber-physical system}{ Postdoctoral research}{}{}{
	\begin{itemize}
		\item[--] Designed BF-IoT - a Wireless fingerprint-based authentication mechanism in IoT.
		\item[--] Securing privacy leakage in Bluetooth based IoT network.
		\item[--] Design authentication mechanisms in Voice based personal assistant such as Alexa.
	\end{itemize}
}

\cventry{2013-2017}{Jamming Avoidance in Dynamic Spectrum Access Networks}{PhD Dissertation}{}{}
  {
  \begin{itemize}
    \item[--] Explored honeynet-based defense mechanism for cognitive radio networks under jamming attack.
    \item[--] Designed stochastic learning mechanism to perceive attacker's strategy by wireless fingerprint. 
    \item[--] Formulated queue model with fixed vacation to analyze traffic behavior of cognitive radios.
  \end{itemize}
  The effectiveness of the proposed mechanism has been evaluated on a state-of-the-art high spectrum agile radio testbed comprising several USRP software defined radios which are controlled using open source GNURadio.
  }

\cventry{2015-present}{Neighbor Discovery in Directional Communication}{Lead Student}{}{}
  {
  \begin{itemize}
    \item[--] Proposed line-of-sight (LOS) discovery methods for mobile nodes with full directional transceivers. Works both for directional RF and free-space-optics (FSO) networks.
    \item[--] Modified helix equations to optimize neighbor discovery scanning in 3D. 
  \end{itemize}
  This work has been evaluated with a prototype built with an off-the-shelf robot car, IR transceivers and Raspberry Pi as the controller. In addition, one patent application has been filed in collaboration with M Khan and Dr. M Yuksel.
  }



\cventry{2015-2016}{Jamming Avoidance Multihop 3D UAS Mesh Networks}{Lead Student}{}{}
  {
  \begin{itemize}
    \item[--] Examined the performance of adaptive beam nulling as a mitigation technique against jamming attacks in multihop ad hoc networks. 
    \item[--] Proposed Kalman filter based tracking model to predict movement of the jammer with discrete DoA estimation; and optimized the beam null region based on stochastic model.
    \item[--] Built 3D mobility model for UAVs in ns3.
  \end{itemize}
  }


\cventry{2010-2011}{Enhance Performance of Voice Service for Congested Networks}{Research Fellow}{}{}
  {
  \begin{itemize}
    \item[--] Proposed optimization mechanism for SIP based VoIP service over wireless networks.
    \item[--] Proposed effective adaptive jitter playout buffer algorithm for real-time application.
    \item[--] Built MAC layer for cognitive radio in OPNET using Proto-C.
    \item[--] Designed cross Layer optimization model of VoIP end-to-end QoS over cognitive radio.
  \end{itemize}
  }


\cventry{2009-2010}{Real Time Traffic over WiMAX}{Masters Dissertation}{}{}
  {
  \begin{itemize}
    \item[--] Evaluated performance of WiMAX under mobility, pathloss models, traffic type and scheduling.
    \item[--] Investigated VoIP service in WiMAX, WiFi and integrated WiMAX-WiFi networks.
    \item[--] Reduced hand-off delay for WiMAX network  by integrating  MAC and mobile-IP.
  \end{itemize}
  }


\section{Testbed Development and Demonstration}
\cventry{2015-present}{Dynamic Spectrum Access Testbed with GNURadio}{}{}{}
  {
  \begin{itemize}
    \item[--] Implemented frequency agile cognitive radio testbed using USRP, GNU Radio.
    \item[--] Inspected performance for channel aggregation, fragmentation, jamming attacks etc.
    \item[--] Implemented full duplex transmission using single radio device.
    \item[--] Implemented dynamic spectrum selection in multi hop mesh networks.
  \end{itemize}
  }

\cventry{2013-2014}{Mobile Frequency agile Testbed}{}{}{}
  {
  \begin{itemize}
    \item[--] Built on top of Atheros chips and ath5k as the WiFi driver.
    \item[--] Investigated pseudo random channel hopping to mitigate jamming.
  \end{itemize}
  }





\section{Student Advisement}
\subsection{Masters Thesis Advisor}
\cventry{2022--2023}{Jesica Vasquez}{MS student}{ Miami University}{}{Topic: Secure Neighbor Discovery Protocol for Free Space Optical Communication Netowrk}

\cventry{2022--2023}{Nick Perry}{MS student}{ Miami University}{}{Topic: Crossfire Attack Detection in 6G Mobile EdgeComputing}

\cventry{2021--2023}{Monu Chaudhary}{MS student}{ Miami University}{}{Topic: NFT Based Fan Card For Athelets}

\cventry{2020--2022}{Angela Famera}{MS student}{Miami University}{}{ Topic: Bio-Inspired Botnet Detection
}

\subsection{Undergrad Student Researcher}
\cvitem{2021--2022}{Sam Kuhbander }
\cvitem{2022--2022}{Daniel Cruz }
\cvitem{2020--2021}{Sara Grimes}
\cvitem{2020--2021}{Nam Hoang}
\cvitem{2021--2011}{Huy Nguyen}
\cvitem{2020--2021}{Yicheng Qian}
\cvitem{2020--2021}{Ryan Schuerkamp}
\cvitem{2020--2021}{Tom Deep}

\subsection{Senior Capstone projects}
\cvitem{2022--2023}{Close bid Auction using Blockchain for Ambulance Hiring}
\cvitem{2022--2023}{Peer to Peer Review System using Blockchain}
\cvitem{2021--2022}{Blockchain based First Responder Application}
\cvitem{2020--2021}{Distributed Storage for First Responders Network}
\cvitem{2020--2021}{Digital Triage Management Application}




\section{International Professional Activities}
\subsection{Session Chair at International Conferences}
\cvitem{2022}{19th IEEEAnnual Consumer Communications \& Networking Conference (CCNC 2022), Las Vegas, USA, session: “WiP5: Autonomous vehicles”}
\cvitem{2015}{International Symposium on Cyberspace Safety and Security  (IEEE CSS) 2015, New York, USA. Track -  Active Defense Techniques and Systems}

\subsection{Member of Technical Program Committee}
\cvitem{CCNC}{IEEE Consumer Communications \& Networking Conference -- 2021, 2022}
\cvitem{ICDCN}{International Conference on Distributed Computing and Networking -- 2022}
\cvitem{MILCOM}{Military Communications Conference -- 2017, 2018, 2019, 2021, 2022
}
\cvitem{ASONAM}{Advances in Social Networks Analysis and Mining -- 2021}
\cvitem{MASS}{IEEE International Conference on Mobile Ad-Hoc and Smart Systems -- 2020}
\cvitem{GameSec}{Conference on Decision and Game Theory for Security -- 2017
}
\cvitem{ICIT}{International Conference on Information Technology -- 2015, 2016}
\subsection{Selected Journal/Conference Reviewer}
\cvitem{Journals}{Computer Communications (Elsevier), Physical Communication (Elsevier), Pervasive and Mobile Computing (Elsevier),  Future Generation Computer Systems (Elsevier),  International Journal of Communication Systems (Wiley), Wireless Communications and Mobile Computing (Wiley), International Journal of Distributed Sensor Networks (Hindawi)}

\cvitem{Conferences}{IEEE INFOCOM, IEEE Globecom, IEEE ICC, IEEE MILCOM, ISCIT, IEEE WoWMoM }





\section{Leadership and Committee Experience}
\cvitem{2021--present}{Advisor to Miami University Cybersecurity Club. Formed with students from my classes. }
\cvitem{2021--present}{Member of Cybersecurity Committee at the department of CSE.}
\cvitem{2014-2017}{Elected thrice as a college of engineering representative at UNR Graduate Student Association.}
\cvitem{2015-2017}{Elected twice as the chair of the  clubs and organizations committee of UNR GSA}
\cvitem{2014-2017}{Serving budget committee of UNR GSA}
\cvitem{2014-2017}{Serving judicial committee of UNR GSA}
\cvitem{2014-2015}{Founding vice president of Computer Science Graduate Student Club, UNR}
\cvitem{2013-2015}{Elected twice as the vice president of Indian Student Organization, UNR}



\section{References}
\cventry{PhD Advisor}
	{Dr. Shamik Sengupta}
	{Ralph E.\& Rose A. Hoeper Professor and Executive Director of UNR Cybersecurity Center}
	{ Department of Computer Science and Engineering} 
	{University of Nevada, Reno. \url{http://www.cse.unr.edu/~shamik/}}
	{Email: ssengupta@unr.edu, Tel: (775) 784-6953}

\cventry{Thesis Committee Member}
	{Dr. Murat Yuksel}
	{Associate Professor} 
	{Department of Electrical and Computer Engineering} 
	{University of Central Florida (UCF), \url{http://www.ece.ucf.edu/~yuksem/}}
	{Email: murat.yuksel@ucf.edu, Tel: (407) 823-4181}
	
\cventry{Department Chair}
	{Dr. Eric R. Bachmann}
	{ Professor} 
	{Department of Computer Science and Software Engineering}
	{Miami University, Ohio}
	{Email: bachmaer@miamioh.edu, Tel: (513) 529-0786}
	
\cventry{Collaborator}
	{Dr. Arthur Carvalho} 
	{Dinesh \& Ila Paliwal Innovation Chair \& Associate Professor} 
	{Department of Information Systems \& Analytics}
	{Miami University, \url{https://arthurcarvalho. info/}}
	{Email: carvalag@miamioh.edu, Tel: (513) 529-7162}


\end{document}

%%% Local Variables:
%%% mode: latex
%%% TeX-master: t
%%% End:
